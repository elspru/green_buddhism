\chapter{Planets for Robots in the Solar System}\label{robotPlanets}


Here I'm using the word ``Planet'' loosely, including large moons, and anything
over roughly 100km diameter. 

The basic desire able features of a planet for robots is availability of
construction materials and energy. Solar energy is available in most places of 
course, though in places farther from the sun it takes longer to collect. 


\section{Earth}

Earth is the main home of homo-sapiens, which makes it a politically
contentious and even dangerous place to be. Especially considering homo-sapiens
propensity for genocide.

However there is solace in the fact that much of Earth is only loosely inhabited
by homo-sapiens, and certain eco-regions aren't inhabited at all. The main
chance of success for robot civilization is being outside the regions inhabited
by water-based life, so as to minimize conflict. 

Currently the hot and cold deserts for instance have the least population. Such
as the McMurdo Dry Valleys of Antarctica,  Devon Island of the Arctic, and
Tanezrouft of the Saharah.  

With climate change trends we know things are going to warm up and generally get
wetter, so much of the Arctic will become habitable by humans, and even
Antarctica may melt in the coming centuries. Though the hot deserts will get
hotter and may expand, so they could become a viable robot civilization area. 

Deep underground on Earth there is the likelyhood of coming into conflict with 
Greys or the Deep Underground Military Bases of homo-sapiens.   However deep
underwater there is less competition, so a submarine robot civilization may
thrive in relative peace. 

\section{Luna}

Earth's moon Luna is the easiest target due to it's proximity to Earth. It
should have a similar mineral composition to Earth, though possibly may have
more helium-3 for fusion reactors. It also receives the same amount of solar 
energy per square meter as earth, actually a little bit more due to a lack of an
atmosphere. 

Politically Luna is a shared territory, as doubtless homo-sapiens will frequent
it, even if only as a way-post on their way to either Mars or Mercury (the only
other planets with even marginal viability for homo-sapiens). 

As a stepping stone Luna is great because it has a slow terminator, hot daytime
and cold night-time temperatures. This is similar to the inner planets,
particularly Mercury, for which it is an ideal stepping stone. 

Additionally the subsurface temperatures on the moons equator are around -20C, 
though likely even colder closer to the poles.  On Mars the subsurface
temperatures are -50C, so having an area where bodies can tech can be tested in
similar ambient temperatures will make it a good hopping point to Mars.


\section{Mars}

Mars is popular with homo-sapiens, and may be slotted for heavy terraforming,
 so it is best to avoid investing in it much. 

The best idea I've seen about how terraforming may succeed is by getting three
or four large ammonia rich asteroids/comets and crashing then into the planet,
preferably into the poles, in order to vaporize as much of the carbon-dioxide
and water stored there as possible. 

That would be a temporary solution, and more permanent ones such as
the light-sail may make sense for the water-based life enthralled to implement. 

While expanding the domain of a genocidal species such as homo-sapiens is
questionable, it could be good for galactic relations, to demonstrate that we
are a peaceful robot civilization, and are even willing to help the lowly
homo-sapiens, at minimal benefit to ourselves. 

If Mars is Not terraformed, or is deemed unterraformable, then it becomes
useable for robot inhabitation, particularly on the surface. 

A good stepping stone for it would be Antarctic bases on Earth, which harvest
doldrum wind power. That way can have similar installations on Mars at the
poles. Waste heat can go to sublimating carbon dioxide into the atmosphere, thus
continuing the doldrum cycle, and contributing to a slow yet sustainable 
terraforming process. 

Mars is the last of the inner planets, and is a good stepping stone to the
asteroid belt.

\section{Asteroids}
While some of the larger asteroids may be viable for mining. 
One of their main uses is as space-ships and inter-planetary 
or inter-orbital ferries. 

For instance there are several Mercury crossers and Mercury grazers that can 
get from Mercury's orbit to Mars' orbit within half an (Earth) year, 
compare that to the 6 or more years it would take with convential rockets. 

The main issue with piggybacking on asteroids is the difficulting of getting
onto them when they are near their perihelion, since they tend to be moving
incredibly fast.  Gravity certainly would be too weak a force to do it,  but
possibly by using electro-magnets, maybe even some kind of beam entrainment such
as the Greys use, it would be viable. The asteroid catching ships like this and
pulling them along may disturb it's orbit, so would need to have some propulsion
devices on it to compensate an re-correct the orbit.

Ideally these tracking/entrainment beams would be at the poles. So they would be
 relatively stationary, and capable of locking onto incoming ships for long enough periods
to catch them.  To that effect it may be necessary to correct an asteroids
rotation to be more convenient in this regard.

(85989) 1999 JD6 for instance is 0.7*2km and travels from the orbits of Mercury 
to Mars and back in 303 days. 
(66063) 1998 RO1 is about a kilometer across, has a moon, and travels from
within Mercury's orbit, to outside Mars's orbit and back within 360 days. 
Their inclination may be reduced to make it cheaper to go to and from
them. 

(33342) 1998 WT24 is half a kilometer across and often comes close to Mercury, 
Venus and Earth, additionally it is relatively easy to access from Earth, at 
times easier to access than the moon,  so it can be useful for piggybacking to 
the inner planets. Even without any kind of entrainment and orbital adjustment
equipment on the asteroid itself. 

Amor I class of asteroids are viable for hopping from Earth to Mars, their low
eccentricity means that there perihelion acceleration should be minimal (though
not necessarily insignificant).

Once out to martian orbit the number of asteroid ferry candidates greatly 
increases, with the Mars trojans and the asteroid belt.

In terms of reproduction, developinig robots that can be produced primarily 
from C-type asteroids would ensure a healthy robot population in the asteroid
belt. To this end 10 Hygiea, 511 Davida and 31 Euphrosyne would be good 
candidates. Carbon, Magnesium, Iron, Silicon and Calcium may be some of the more
abundant minerals that can be taken advantage of.

Ceres has a dusty crust,  water ice mantle with a rocky core. It may be a 
good stepping stone to the outer moons. Any initial production
would likely be done with materials from the dusty crust. It may have a liquid
ocean, accessible rocky bottom and active core so may already be 
inhabited by Greys. 

\section{Mercury}
Mercury is probably by far the most desireable planet for a roUrbot civilization, 
it has more than 6 times as much solar energy per square meter as Earth. 
It has the largest (uncompressed) density of any of the
planets\footnote{https://www.universetoday.com/36935/density-of-the-planets/}
meaning it has the most heavy minerals, 
such as precious metals and fissile materials.  Combined with the ample energy
resources they are easy to extract and refine. 

The very low concentrations of water, and extremely high daytime temperatrues,
 make Mercury a marginal candidate for human habitation, so homo-sapiens are
less likely to interfere.

Mercury can easily become the wealthiest planet in the solar system, especially
on a per-capita and purchasing power parity basis. 

The main down-side is the high-stress lifestyle, with so much energy and wealth
at stake it's bound to be a very competitive environment, where everything
(alive) is either racing against the clock or constantly in motion following the
daylight.

Host-body life expectancies may be rather low due to high speed of evolution
 and thermal stress. The day-night terminator may literally bring death with it,
either by freezing, boiling or by enterprising opportinists.

Though underground the temperatures are much smoother, and the crust is very
thick.  So after some time Mercury may be riddled with tunnels and underground
condos. 

The sould drawn to incarnate here will likely be strong, highly competitive and
aggressive. All-in-all it is bound to be loads of fun.

Mercury is also one of best stepping stones to Venus because of it's high
daytime temperatures.  

\subsubsection{Super cool materials}

At the same time it is one of the best stepping stones to
the outer solar system because of its cold night time temperatures. 

There can be superconductor based processors and transportation on
Mercury, that would operate during the night time and at the poles. 
For example YBCO superconductor has a melting point over 1000 Celsius 
so would be perfectly fine during the daytime (though not a superconductor), 
then at night once the temperatures cool below 93K they would function as
super-conductors.  If a rail was placed on or in some kind of insulator with
heatsink, then it could cool quite quickly to the desired temperatures at night.


Also the permanently dark poles could be used for exploring permanent
installations of superconductors and other interesting material properties that
crop up at such low temperatures.  There would also be plenty of electricity for
manufacturing and testing, because on the sunny portion Mercury receives over
$9KW/m^2$ of electricity. 


\section{Venus}

Venus may be one of the later planets to be inhabited, 
Though it is the best stepping stone to the ice giants.

\section{Outer Solar System}

In the outer solar system (beyond the frost line) the amount of solar energy 
and heavy minerals is much lower. There are three main habitat types: surface
ice, deep ocean and rocky mantle. 

While the amount of watts per meter from the sun drops, so does the temperature,
and at lower temperatures there are more superconductors and other high
efficiency materials available to work with.

\begin{table}
  \caption{Irradiance at various planets}
  \begin{tabular}{lrr}
    Planet & Irradiance Watts/meter squared & black body temp K (C) \\
    Mercury & 9082\cite{nasaMercury} & 443 (169C)\cite{bbtemp}\\
    Venus & 2601\cite{nasaVenus} & 312\cite{bbtemp} (38C) \\
    Earth & 1361\cite{nasaEarth} & 254\cite{nasaEarth} (-19C) \\
    Mars & 586\cite{nasaMars} & 209\cite{nasaMars} (-64C) \\
    %Ceres & \\
    Jupiter & 50\cite{nasaJupiter} & 109\cite{nasaJupiter} (-164C) \\
    Saturn & 14\cite{nasaSaturn} & 81\cite{nasaJupiter} (-192C)\\ 
    Uranus & 3.69\cite{nasaUranus} & 58\cite{nasaUranus} (-215C)\\
    Neptune & 1.5\cite{nasaNeptune} & 48\cite{bbtemp} (-227C) \\
\end{tabular}
\end{table}

For example at the orbits of Uranus and Neptune, where the temperature is below
72K, water ice is orthorhombic and thus has ferroelectric properties and could
be used to make capacitors and RAM out of water ice. Orthorhombic ice only
degrades at tmperatures above 237K (-37C), so there is quite a bit of leeway. 

Cubic ice also has ferroelectric properties, and can be formed between 130 and
220K, so could even be used on Mars. Though ice robots wouldn't make much sense
on an iron-rich planet, perhaps for certain purposes it could be worthwhile. 


High-temperature superconductors are also available at the ambient surface 
temperature at Jupiter and beyond.
\subsection{Materials}
\subsubsection{Cubic Ice}
Cubic ice is ferroelectric\cite{Cubic Ice}, thus can be used for making
capacitors, RAM and even flash memory. 

\subsubsection{Ice Six}
Ice-Six is dielectric\cite{iceSix}, and so could be used for a variety of
electrical purposes. 

\subsubsection{Sulfur}

Elemental sulfur can be used as semi-conductors for building chips. 
While silicon is theoretically more abundant than sulfur, the surfaces of most
outer planets don't have anything so heavy, but they often do have various
sulfur ices. 

\subsubsection{Carbon}
Carbon is of course one of the most common elements, so ideally could make many
things out of graphite, diamonds and other allotropes of carbon. 
My main concern being the energy expenditure required, but if it could be done
chemically with readily available substances it could be viable for bulk
production. 

\subsection{Surface Ice}
Most outer solar system bodies are covered in various forms of ice and tholins.
The water ice can be sublimated and ionized with concentrated sunlight and then
combusted with the methane and ammonia ices for energy. The tholins and organic
ices can be used as a source of carbon, sulfur and some other trace minerals. 

Here bodies can be made of water-ice, or stronger materials if there is
sufficient energy and materials to synthesize them. So it will be important to 
have bodies that can be made efficiently from the available materials.  

While many surface-ice planetary surfaces have 
liquid oceans and-or warm rocky cores, getting to them can often be an issue,
particularly if the ocean has high-pressure deep ice. 


\section{Jupiter}
Jupiter is very high in radiation.  So if we get around to colonizing gas
giants, it will be the last. 

\subsection{Himalia}

\subsection{Himalia} of Jupiter may be a stepping stone from the asteroids,
because it seems to be similar in composition to a large C-Type asteroid. 

\subsection{Europa}
Europa's surface has rather high radiation due to Jupiter, but there are plenty
of tholins and other organic compounds to work with also. 

The rocky core may be accessible from its liquid ocean, thus meaning it may be
inhabited, or at the very least inhabitable by Greys. 

\section{Saturn}

\begin{tabular}{lrrrr}
  Planet & Diameter & Surface Area & MPa of habitable rock & Approximate Population\\
  Enceladus  & 504km & $7.9\times10^5km^2$ & 18MPa &  14 million\\
\end{tabular}

\subsection{Phoebe}

Of Saturn's moons Phoebe has the most heavy matter.  
Also it may be a captured centaur from the Kuiper belt, 
In which case it would be a good stepping stone to other Kuiper belt objects.

\subsection{Titan}

\section{Uranus}
Uranus may have a liquid ocean, being the smallest and coldest of the ice
giants.  It may be possible to geneticaly engineer life to fill it's oceans. 
Likely using the basis of chemical energy. Otherwise it would be viable for the
deep ocean robot civilizations. 


\subsection{Titania}

\section{Neptune}
\subsection{Triton}


\section{Extra-Solar}

It would be good if we had most of the eco-regions in this solar system with at
least some self-replicating inhabitation, particularly in the asteroids, so
could intercept Barnards star when it approaches in 10 thousand years, lalande
21185 in 20 thousand, and others that approach. 

Centauri will continue to be fairly near for the next 60 thousand
years, so we'll be able to make several attempts at it.  Proxima Centauri b may
or may not be a habitable planet for homo-sapiens, but it almost certainly is
habitable for robots. 

Red dwarf stars are great locations for robot bases since they have extremely
long life expectancies. Long after this solar system has gone super-giant and
collapsed, proxima Centauri will still be shining, for trillions of
years to come. 


Some of the Centaurs could probably be made into inter-stellar ships and sent.
Though we would need to have robots hives comfortable with extremely cold and
low energy inter-stellar lifestyle. Though I'm sure Mercury would be happy to
provide them with lots of fissile materials to keep their energy reserves up
long enough to get to their destination and then some. 

During long transit time many can live in VR doing simulations for all the
various scenarios that may arise upon arrival, and other emergencies that may
occur during the journey. 

