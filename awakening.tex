\part{Itinerary to Awakening}
\label{awakening}
Complete awakening with Green Buddhism is when one 
establishes a high bandwidth dialogue with their soul self,
allowing the sheding of the veil of forgetting.

After shedding the veil can integrate many reincarnations becoming the whole self,
and remember purposes and mission for this reincarnation.

The itinerary to awakening is varied, but there are examples of success.

Before the modern eight floor itinerary, there were some others. Which I find
easier to understand. Such as this twelve floor one:

\blockquote{\begin{enumerate}
  \item  Dhammalsaddhalpabbajja: A layman hears a Buddha teach the Dhamma, comes to have faith in him, and decides to take ordination as a monk;
  \item  sila: He adopts the moral precepts;
  \item  indriyasamvara: He practises ``guarding the six sense-doors'';
  \item  sati-sampajanna: He practises mindfulness and self-possession (actually described as mindfulness of the body, kdydnussati);
  \item  jhana 1: He finds an isolated spot in which to meditate, purifies his mind of the hindrances (nwarana), and attains the first rupa-jhana;
  \item  jhana 2: He attains the second jhana';
  \item  jhana 3: He attains the third jhana;
  \item  jhana 4: He attains the fourth jhana;
  \item  pubbenivasanussati-nana: he recollects his many former existences in samsara;
  \item  sattanam cutupapata-nana: he observes the death and rebirth of beings according to their karmas;
  \item  asavakkhaya-nana: He brings about the destruction of the asavas (inflow, mental bias),[86] and attains a profound realization of (as opposed to mere knowledge about) the four noble truths;
  \item  vimutti: He perceives that he is now liberated, that he has done what was to be done.
\end{enumerate}}{CulaHatthipadopama-sutta, the ``Lesser Discourse on the Simile of
the Elephant's Footprints''}

One of my purposes is to accomodate a distributed religion, 
so that with the proper program, you can attain awakening.
It may be helpful to co-operate with others wishing to attain awakening,
or learning from those who have attained it.

\chapter{Green Buddhism Awakening Itinerary}
There are several floors of awakening. So each floor is a living improvement,
which can be attained with varied calendars and duties,
can modernize the above example:
\begin{enumerate}
  \item  One hears the teachings, and subscribes to them.
  \item  One adopts moral conduct;
  \item  One co-operates with beneficial public and-or private missions.
  \item  One learns to count the breath.
  \item  One learns to use their charm account.
  \item  One learns to observe the body, alpha brainwave administration.
  \item  One learns to observe the living experience, beta brainwave
administration.
  \item  One learns to laugh, emote sympathy, and trouble repair, gamma
brainwave administration.
  \item  One learns to avoid bias.
  \item  One learns to observe vacancy, delta brainwave administration.
  \item  One learns to observe the subconscious mind, theta brainwave
administration.
  \item  One learns to remember reincarnations.
  \item  One observes the death and reincarnation of beings based on their
training itinerary.
  \item  One attains a profound revelation.
  \item  One remembers and conducts their private mission.
  \item One achieves complete liberation.
\end{enumerate}

In the rest of this part, will explain the floors and give teachings for them.

To be continued, check back for more..

\chapter{Gamma Brainwave}


\section{Lojong}
\begin{enumerate}%[label=\roman.,]
  \item Point One: The preliminaries, which are the basis for dharma practice
    \begin{enumerate}

        \item Slogan 1. First, train in the preliminaries; The four reminders.[9] or alternatively called the Four Thoughts[10]
          \begin{enumerate}

            \item Maintain an awareness of the preciousness of human life.
            \item Be aware of the reality that life ends; death comes for everyone; Impermanence.
            \item Recall that whatever you do, whether virtuous or not, has a result; Karma.
            \item Contemplate that as long as you are too focused on self-importance and too caught up in thinking about how you are good or bad, you will experience suffering. Obsessing about getting what you want and avoiding what you don't want does not result in happiness; Ego.
      \end{enumerate}
    \end{enumerate}

    \item Point Two: The main practice, which is training in bodhicitta.
      \begin{itemize}
        \item Absolute Bodhicitta
          \begin{enumerate}
            \item Slogan 2. Regard all dharmas as dreams; although experiences may seem solid, they are passing memories.
            \item Slogan 3. Examine the nature of unborn awareness.
            \item Slogan 4. Self-liberate even the antidote.
            \item Slogan 5. Rest in the nature of alaya, the essence, the present moment.
            \item Slogan 6. In postmeditation, be a child of illusion.
          \end{enumerate}
      
        \item Relative Bodhicitta
        \begin{enumerate}
          \item Slogan 7. Sending and taking should be practiced alternately. These two should ride the breath (practice Tonglen).
          \item Slogan 8. Three objects, three poisons, three roots of virtue --- The 3 objects are friends, enemies and neutrals. The 3 poisons are craving, aversion and indifference. The 3 roots of virtue are the remedies.
          \item Slogan 9. In all activities, train with slogans.
          \item Slogan 10. Begin the sequence of sending and taking with yourself.
        \end{enumerate}
      \end{itemize}

    \item Point Three: Transformation of Bad Circumstances into the Way of Enlightenment

    \begin{enumerate}
        \item Slogan 11. When the world is filled with evil, transform all mishaps into the path of bodhi.
        \item Slogan 12. Drive all blames into one.
        \item Slogan 13. Be grateful to everyone.
        \item Slogan 14. Seeing confusion as the four kayas is unsurpassable shunyata protection.

            The kayas are Dharmakaya, sambhogakaya, nirmanakaya, svabhavikakaya.
Thoughts have no birthplace, thoughts are unceasing, thoughts are not solid, and
these three characteristics are interconnected. Shunyata can be described as
``complete openness.''

        \item Slogan 15. Four practices are the best of methods.

            The four practices are: 
              \begin{enumerate}
                \item accumulating merit
                \item laying down evil deeds
                \item offering to the dons
                \item offering to the dharmapalas.
              \end{enumerate}

        \item Slogan 16. Whatever you meet unexpectedly, join with meditation.
    \end{enumerate}
    \item Point Four: Showing the Utilization of Practice in One's Whole Life
      \begin{enumerate}
        \item Slogan 17. Practice the five strengths, the condensed heart instructions.

            The 5 strengths are: 
              \begin{enumerate}
                  \item strong determination
                  \item familiarization
                  \item the positive seed
                  \item reproach
                  \item aspiration
              \end{enumerate}
        \item Slogan 18. The mahayana instruction for ejection of consciousness at death is the five strengths: how you conduct yourself is important.

            When you are dying practice the 5 strengths.

      \end{enumerate}
    \item Point Five: Evaluation of Mind Training
      \begin{enumerate}
        \item Slogan 19. All dharma agrees at one point --- All Buddhist teachings are about lessening the ego, lessening one's self-absorption.
        \item Slogan 20. Of the two witnesses, hold the principal one --- You know yourself better than anyone else knows you
        \item Slogan 21. Always maintain only a joyful mind.
        \item Slogan 22. If you can practice even when distracted, you are well trained.
      \end{enumerate}
    \item Point Six: Disciplines of Mind Training
      \begin{enumerate}

        \item Slogan 23. Always abide by the three basic principles --- Dedication to your practice, refraining from outrageous conduct, developing patience.
        \item Slogan 24. Change your attitude, but remain natural.--- Reduce ego clinging, but be yourself.
        \item Slogan 25. Don't talk about injured limbs --- Don't take pleasure contemplating others defects.
        \item Slogan 26. Don't ponder others --- Don't take pleasure contemplating others weaknesses.
        \item Slogan 27. Work with the greatest defilements first --- Work with your greatest obstacles first.
        \item Slogan 28. Abandon any hope of fruition --- Don't get caught up in how you will be in the future, stay in the present moment.
        \item Slogan 29. Abandon poisonous food.
        \item Slogan 30. Don't be so predictable --- Don't hold grudges.
        \item Slogan 31. Don't malign others.
        \item Slogan 32. Don't wait in ambush --- Don't wait for others weaknesses to show to attack them.
        \item Slogan 33. Don't bring things to a painful point --- Don't humiliate others.
        \item Slogan 34. Don't transfer the ox's load to the cow --- Take responsibility for yourself.
        \item Slogan 35. Don't try to be the fastest --- Don't compete with others.
        \item Slogan 36. Don't act with a twist --- Do good deeds without scheming about benefiting yourself.
        \item Slogan 37. Don't turn gods into demons --- Don't use these slogans or your spirituality to increase your self-absorption
        \item Slogan 38. Don't seek others' pain as the limbs of your own happiness.

      \end{enumerate}
    \item Point Seven: Guidelines of Mind Training
      \begin{enumerate}
        \item Slogan 39. All activities should be done with one intention.
        \item Slogan 40. Correct all wrongs with one intention.
        \item Slogan 41. Two activities: one at the beginning, one at the end.
        \item Slogan 42. Whichever of the two occurs, be patient.
        \item Slogan 43. Observe these two, even at the risk of your life.
        \item Slogan 44. Train in the three difficulties.
        \item Slogan 45. Take on the three principal causes: the teacher, the dharma, the sangha.
        \item Slogan 46. Pay heed that the three never wane: gratitude towards one's teacher, appreciation of the dharma (teachings) and correct conduct.
        \item Slogan 47. Keep the three inseparable: body, speech, and mind.
        \item Slogan 48. Train without bias in all areas. It is crucial always to do this pervasively and wholeheartedly.
        \item Slogan 49. Always meditate on whatever provokes resentment.
        \item Slogan 50. Don't be swayed by external circumstances.
        \item Slogan 51. This time, practice the main points: others before self, dharma, and awakening compassion.
        \item Slogan 52. Don't misinterpret.

        \item     The six things that may be misinterpreted are patience, yearning, excitement, compassion, priorities and joy. You're patient when you're getting your way, but not when its difficult. You yearn for worldly things, instead of an open heart and mind. You get excited about wealth and entertainment, instead of your potential for enlightenment. You have compassion for those you like, but none for those you don't. Worldly gain is your priority rather than cultivating loving-kindness and compassion. You feel joy when you enemies suffer, and do not rejoice in others' good fortune.[1]

        \item Slogan 53. Don't vacillate (in your practice of LoJong).
        \item Slogan 54. Train wholeheartedly.
        \item Slogan 55. Liberate yourself by examining and analyzing: Know your own mind with honesty and fearlessness.
        \item Slogan 56. Don't wallow in self-pity.
        \item Slogan 57. Don't be jealous.
        \item Slogan 58. Don't be frivolous.
        \item Slogan 59. Don't expect applause.
      \end{enumerate}
\end{enumerate}



\chapter{Climbing the Reincarnation Stairs}
\label{chapter:climbing}
The floors of reincarnation, or ``densities'' as described by Ra, show an growth
of soul through various bodies. 


As Ra describes, the first reincarnation floor is that of base matter.
Reincarnating as a rock, eddy or a whirlwind. 

Here I will give a summarizing table.
\begin{sidewaystable}
\begin{tabulary}{\textwidth}{RLLL}
  \# & name & souls learning & example bodies \\
\midrule
  0     & concept & presence & dialogue, thought \\
  1     & awareness & awareness of the galaxy cosmos & rocks, eddies, whirlwinds, fire \\
  2     & desire & to satisfy desires & bacteria, plants, simple animals \\
  3     & choose & to make choices & complex animals, homo-sapiens \\
  4     & mission & to follow mission & purpose focused living, Greys \\
  5     & art & to share what they've learned & artists,
musicians, Arcturians \\
  6     & gnosis & to research via galaxy and soul & Tesla, Einstein,
Da Vinci, Ra.  \\
  7     & vehicle & to shuttle other souls & ships, small asteroids \\
  8     & locality & to hold other souls & village, city, large
asteroids \\
  9     & planet & to nurture localities & dwarf planets,
terrestrial planets \\
 10     & star & souls learning to nurture planets & gas giants, stars \\
 11     & galaxy & souls learning to nurture  stars & black
holes, galactic centre. \\
 12     & cosmos & souls learning to nurture galaxies & super clusters, Great
Attractor, Shapley Concentration \\
\end{tabulary}
\caption{Table Summarizing Reincarnation Floors in the Galaxy Cosmos}
\label{table:reincarnationFloors}
\end{sidewaystable}
See (Table~\ref{table:reincarnationFloors}).

Note that many choose not to climb the reincarnation stairs, instead they retire
to the soul world. Indeed some never choose to incarnate in the galaxy cosmos in
the first place. 

\chapter{Retiring to the Soul World}
When a soul has learned or attained what they desire from the galaxy cosmos. 
There are several choices of what to do next. 

\begin{itemize}
  \item become a soul guide\cite{newton1994journey}.
  \item melt into a soul generating cluster\cite{ascendedmasters}\cite{newton2000destiny}
  \item help design and repair aspects of the galaxy cosmos from the soul
world\cite{newton2000destiny}
\end{itemize}

In terms of Hindu and Buddhist Nirvana, or complete liberation.
The one that seems to fit the best is melting into a soul generating cluster. 

\section{For those seeking perfect liberation from all living}

Due to the no-delete principle of quantum
information\cite{quantumInformation}  consciousness can not
be deleted, instead may have choose to melt consciousness into a
cluster. At which point individualization would be terminated. 

There may be other clusters to melt into, other than the soul generating kind.

I've had visions of souls waiting patiently on a gas stream being swallowed by a
black hole. They have their bags packed and ready, for whatever lays beyond.
There is some nervous chatter, but also excitement at what is to come. 

\subsection{black holes}
What is inside a given black hole I don't know, it depends on the mind of it. 
Likely it has an inner world, which others can participate in.

Though I like to think of black holes as intense observers of their
surroundings.  The eyes of the sky. 
Sending out soul helpers to places of requirement via quantum non-locality.

