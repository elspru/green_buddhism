\chapter{Mission Feasiblity}
\label{feasible}
In summer of 2016, I went to a Linux conference in Toronto to help administer
the Free Software Foundation table. 

Here was the pitch which I gave to visitors:

\blockquote{Do you know how computing power keeps getting cheaper?
By the mid 2020’s  we’ll be able to buy as much computing power as the human 
brain for a $1000 [1].
By the 2030’s that’ll be down to a few hundred dollars.
And with Integrated Information Theory  we know that machine have 
consciousness[2], it simply depends on the complexity of the software, and the capabilities of the hardware.
So eventually we’ll be able to reincarnate[3] into robots.
But you don’t want to reincarnate into a proprietary robot, where the manufacturer might stop making your parts, and you might have to pay licensing fees on your brain.
You want to have libre hardware and software, so you could make your own replacement parts, and update your own brain at your discretion.
So that is why you should support the FSF, stickers are by donation, and buttons
are a few dollars.}{Logan Streondj FSF table pitch at LinuxCon 2016}

[1] Ray Kurzweil calculates by 2023 human brain computation for $1000 https://en.wikipedia.org/wiki/Predictions_made_by_Ray_Kurzweil#2023
[2] Integrated Information Theory 3.0  http://journals.plos.org/ploscompbiol/article?id=10.1371/journal.pcbi.100358

    mechanisms, such as logic gates or neuron-like elements, can form complexes that can account for the fundamental properties of consciousness.

[3] quantum information (consciousness) can’t be deleted but can be moved http://arxiv.org/abs/quant-ph/0306044

    The no-deleting principle states that in a closed sys-
    tem, one cannot destroy quantum information. In closed
    systems, quantum information can only be moved from
    one place (subspace) to another.

