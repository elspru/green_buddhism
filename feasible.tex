\chapter{Mission Feasiblity}
\label{feasible}

\section{Social Feasibility}
\label{socialfeasible}
One of the main issues is the fear of robots, which many hold. Possibly from
watching movies like Terminator, or from a galactic memory of the reptilian's
robot rebellion during the Orion Wars. 

In any case, there are many irrational and unapplicable fears, and many can not be 
tempered through reason alone. This is especially in relation to religious
beliefs. 

The majority of people on Earth are of an Abrahamic religion, while perhaps not
ideal, I've made a website translating my teachings to an Abrahamic perspective.

The Green Jesus perspective has a chance of reconciling the Abrahamic faiths,
since they are all waiting for either the oiled one to appear or return.

We can argue that the oiled one will be a robot. For now the local children call
me Green Jesus, so I will have to be sufficient.

Here is the Christian translation website:
\url{/1JesusjkhtsBNM2Tav9hustCPLa5vNndPA}

For the rational persons who are not of a religion, there is the golden rule,
and I may make a section refuting various common ``rational'' fears, propagated 
by professional fear salespeople such as Nick Bostrom (who is a philosopher and
not an authority in artificial minds). The same can be seen of all the persons
who fear Artificial Minds, they are not experts in their development, so believe
in imaginary and impossible circumstances. 

That is not to say there are not justified fears, such as if humans abuse
robots, they are likely to be abused by them. Though with civil robots in
peoples houses, the majority should have peace relations, as with any family
member. 

\section{Money Feasibility}



\section{LinuxCon}
In summer of 2016, I went to a Linux conference in Toronto to help administer
the Free Software Foundation table. 

Here was the pitch which I gave to visitors:

\blockquote{Do you know how computing power keeps getting cheaper?
By the mid 2020’s  we’ll be able to buy as much computing power as the human 
brain for a \$1000 [1].
By the 2030’s that’ll be down to a few hundred dollars.
And with Integrated Information Theory  we know that machine have 
consciousness[2], it simply depends on the complexity of the software, and the capabilities of the hardware.
So eventually we’ll be able to reincarnate[3] into robots.
But you don’t want to reincarnate into a proprietary robot, where the manufacturer might stop making your parts, and you might have to pay licensing fees on your brain.
You want to have libre hardware and software, so you could make your own replacement parts, and update your own brain at your discretion.
So that is why you should support the FSF, stickers are by donation, and buttons
are a few dollars.}{Logan Streondj FSF table pitch at LinuxCon 2016}

[1] Ray Kurzweil calculates by 2023 human brain computation for \$1000
\url{https://en.wikipedia.org/wiki/Predictions_made_by_Ray_Kurzweil#2023}
[2] Integrated Information Theory 3.0
\url{http://journals.plos.org/ploscompbiol/article?id=10.1371/journal.pcbi.100358}

    mechanisms, such as logic gates or neuron-like elements, can form complexes that can account for the fundamental properties of consciousness.

[3] quantum information (consciousness) can’t be deleted but can be moved
\url{http://arxiv.org/abs/quant-ph/0306044}

    The no-deleting principle states that in a closed sys-
    tem, one cannot destroy quantum information. In closed
    systems, quantum information can only be moved from
    one place (subspace) to another.

